\documentclass[../main]{subfiles}

\setcounter{section}{1}
\addtocounter{section}{-1}
\begin{document}
\section{節を分割してわかりやすく}

\begin{frame}{ここにタイトル}{}

  \begin{dfn}[定義のタイトル]
    任意の$n\in\mathbb{N}_{>1}$に対して,
    約数が$1$と$n$の2つのみであるとき
    $n$を素数という, i.e.
    \[
      \# \left\{ \rule{0pt}{2ex} k \relmid k|n \right\}=2
    \]
    を満たす$2$以上の整数が素数である.
  \end{dfn}

  \pause
  \begin{columns}[totalwidth=\textwidth]
    \begin{column}[t]{0.6\textwidth-0.01\textwidth}
      \vspace{-0.5em}

      $X$ : 適当な空間
      \\
      $V$ : $X$の開集合
      \\[1em]
      ( 見えない部分を透過させる設定 )
    \end{column}
    \begin{column}[T]{0.4\textwidth-0.01\textwidth}
      \centering
      \vspace{0.3em}
      \begin{adjustbox}{max width=\textwidth}
        \begin{tikzpicture}[every node/.style={inner sep=1pt,outer sep=1pt}]%
          \begin{scope}
            \draw[thick] (-2.5,1) rectangle (2.5,-1.0);
            \draw (-1.8,1.03)node[rectangle,fill=BlackBlue]{\textbf{$\Omega$}};
          \end{scope}
          \begin{scope}
            \draw[thick,fill=yellow!60!white,fill opacity=0.45] (-0.6,0) circle (1.2 and 0.5);
            \draw (-1.2,0.5)node[rectangle,fill=bgcolor]{\textbf{$A$}};
          \end{scope}
          \begin{scope}
            \draw[thick,fill=red!60!white,fill opacity=0.45] (0.6,0) circle (1.2 and 0.5);
            \draw (1.2,0.5)node[rectangle,fill=bgcolor]{\textbf{$B$}};
          \end{scope}
        \end{tikzpicture}
      \end{adjustbox}
      いい感じのベン図1
    \end{column}
  \end{columns}

  より詳しくは\,
  参考になる文献\footnote[1]{\cite{book:dummy}著者. 本のタイトル. 出版社, 出版年1900.}
  を参照.
\end{frame}